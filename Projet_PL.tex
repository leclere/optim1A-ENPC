


\documentclass[10pt]{article}
\usepackage[utf8]{inputenc}
\usepackage{lmodern,textcomp,eurosym}
\usepackage[margin=1.5cm]{geometry}

\usepackage{latexsym}     % Pour certains symboles (\Box, ...)
\usepackage{amsmath}      % Environnement AMS
\usepackage{amsthm}       % avec theoremes
\usepackage{amsfonts}     % et fonts associees
\usepackage{amssymb}      % et symboles
\newcommand{\eur}{\geneuro }
\usepackage{enumitem}

\def\x{\mathbf{x}}
\def\R{\mathbb{R}}
\def\btheta{\boldsymbol{\theta}}

\title{Projet personnel - Introduction à l'optimisation}
\author{vincent.leclere@enpc.fr}

\begin{document}



\maketitle

{\em Les sections sont largement indépendantes et de difficultés croissantes. }

\section{Présentation du problème et théorie}

On suppose qu'un processeur doit traiter $n$ tâches, potentiellement en parallèle. Chaque tâche $i\in[n]$
nécessitant une charge de travail totale $w_i \geq 0$, qui doit être effectuée entre l'instant
$d_i$ et l'instant $f_i$. On note $x_t \geq 0$ la quantité de travail effectuée à la période
$t \in [T]$.
On noteras $y_t^i \geq 0$ la quantité de travail concernant
la tâche $i$ traitée à l'instant $t$. Cette quantité est décidé par le gestionnaire du calculateur.
On note $\varphi(x_t)$ la quantité d'énergie nécessaire au processeur à l'instant
$t$ pour traiter une charge totale $x_t = \sum_{i=1}^N y_t^i$.
Finalement, à chaque instant $t\in[T]$ on souhaite
que le processeur ait une charge de travail $x_t \in [\underline{x},\bar{x}]$.
Dans tout le projet on supposera qu'il existe un planning de charge réalisable $(x_t)_{t\in[T]}$.

\begin{enumerate}
 \item On considère l'ensemble
 $$X = \{x \in [\underline{x},\bar{x}]^T \quad | \quad \exists y \in \R_+^{nT}, \; \forall i\in[n],\; \sum_{t=d_i}^{f_i}y_t^i = w_i\}$$
 Justifier que $X$ est l'ensemble des plannings de charge réalisables. Que pouvez vous dire de sa géométrie ?
 \item \'Ecrire un problème d'optimisation $(P)$ qui a pour but de minimiser l'énergie totale dépensée pour traiter l'ensemble des tâches $i\in[N]$ en temps et en heure.
 \item Si $\varphi$ est convexe différentiable, montrer qu'il s'agit d'un problème d'optimisation convexe différentiable.
 \item Supposons que $\varphi$ soit affine, i.e. $\varphi : x  \mapsto ax + b $ avec $a,b\in\R$.
 Calculer la valeur d'un planning réalisable $x$. Un solveur linéaire est-il utile ?
 \item Si $\varphi$ est un polynôme de degré $2$ donner une condition simple pour que le problème soit convexe.
\end{enumerate}

\section{Résolution du problème à l'aide de Julia}
On considère $N=10$ tâches, sur un horizon de $T=12$ pas de temps. On suppose que $\varphi(x) = \alpha x^2+ \beta x + \gamma $ est un polynôme de degré deux. On choisit les paramètres suivant
\begin{itemize}
	\item $\alpha = 1$, $\beta=1$, $\gamma = 1$
	\item $\mathbf{d}=\textrm{[1 1 5 7 7 8 6 7 8 9]}$
	\item $\mathbf{f}=\textrm{[2 3 5 9 10 8 9 10 12 12]}$
	\item $\underline{x}=1$, $\bar{x} = 10$
	\item $\mathbf{w}=\textrm{[10 10 3 7 10 2 2 2 7.5 7.5]}$
\end{itemize}
\begin{enumerate}[resume]
	\item Modéliser le problème $(P)$ à l'aide de Julia / JuMP et résoudre le problème.
	Calculer un planning de charge optimal, ainsi que la quantité d'énergie minimum $v_P$ pour traiter l'ensemble des tâches.
	\item Pour cette question uniquement on ajoute une contrainte de rampe sur la puissance de calcul, plus précisément on souhaite que la charge de travail totale $x_{t+1}$ à l'instant $t+1$ ne soit pas plus de $1$ au dessus ou en dessous de la charge de travail totale $x_t$ à l'instant $t$ (la charge de travail initiale $x_1$ n'est pas contrainte).
	\begin{enumerate}
		\item quelles contraintes mathématique faut-il ajouter pour représenter cette contrainte de rampe ?
		\item la représenter sous forme de contraintes linéaires, et l'introduire dans le problème JuMP. Calculer un planning de charge optimal ainsi que la nouvelle quantité d'énergie minimum. Comparer à la question précédente.
	\end{enumerate}
\end{enumerate}

%{\bf Nous ne conservons pas la contrainte de rampe pour la suite du projet.}

\section{Un nouvel objectif}
Dans cette section nous cherchons à minimiser la date de fin de traitement de l'ensemble des tâches.
\begin{enumerate}[resume]
	\item Appelons $\tau \in [12]$ la date à laquelle toutes les tâches sont traitées. Proposer une méthode itérative permettant de trouver un planning de charge minimisant $\tau$.
	\item Résoudre avec JuMP ce problème sur les données numériques de la section $2$.
	\item Proposer une démarche permettant de trouver, parmis les planning ayant un $\tau$ minimal, un planning minimisant l'énergie totale dépensée.
\end{enumerate}

\section{Des approximations linéaires de $(P)$}

{\em Cette partie est légèrement plus difficile que les précédentes. Elle permet de dépasser le 15/20.}

Le problème traité est un problème linéaire-quadratique convexe, plus difficile à résoudre qu'un problème linéaire. Nous allons construire des approximations linéaires du problème, que nous pourrons résoudre à l'aide d'un solveur linéaire (Clp). Rappelons que $\varphi:x\mapsto \alpha x^2 + \beta x + \gamma$ est supposée convexe.
\begin{enumerate}[resume]
	\item Montrer que $(P)$ est équivalent à $(P+)$ défini par
		\begin{align*}
		(P+) \qquad \min_{\x \in X, \btheta \in \R^T} \quad & \sum_{t=1}^T \theta_t \\
		\mbox{s.c.} \quad & \theta_t \geq \varphi(x_t)
		\end{align*}


	\item Pour $k\in\mathbb{N}$ fixé, nous allons construire une approximation extérieure $(P^{out}_k)$. Soit $k$ réels $x_i\in [\underline{x},\bar{x}]$, tels que $\underline{x} = x_1 \leq \dots \leq x_k = \bar{x}$. On définit
		$$ \underline{\varphi}^k : x \mapsto \max_{j\in[k]} \; \{\varphi(x^k) + \varphi'(x^k)(x-x^k)\}.$$
		On définit le problème $(P^{out}_k)$ comme
		\begin{align*}
		(P^{out}_k) \qquad \min_{\x \in X, \theta \in \R^T} \quad & \sum_{t=1}^T \theta_t \\
		\mbox{s.c.} \quad & \theta_t \geq \underline{\varphi}^k(x_t)
		\end{align*}
	\begin{enumerate}
		\item Justifier que $(P^{out}_k)$ est un problème linéaire. De plus montrer que si $(\x, \btheta)$ est une solution admissible de $(P^{out}_k)$, alors $\x$ est un planning de charge admissible.
		\item Montrer que, pour tout $x \in \R$, $\varphi(x) \geq \underline{\varphi}^k(x)$.
		\item En déduire que la valeur $v^{out}_k$ du problème $(P^{out}_k)$ est une borne inférieure de $v_P$ la valeur du problème $(P)$.
		\item Comment déduire de la solution de $(P^{out}_k)$ une borne supérieure de $v_P$ ?
	\end{enumerate}
	\item En choisissant $x_i = i$ pour $i\in [10]$ résoudre à l'aide de JuMP le problème $(P^{out}_{10})$. En déduire une borne inférieure et une borne supérieure de $v_P$.
	\item Construisons une approximation intérieure $(P^{in}_k)$.
	\begin{enumerate}
		\item A l'aide de la convexité trouver une fonction linéaire par morceaux $\bar{\varphi}^k$ qui majore $\varphi$ sur $[\underline{x},\bar{x}]$ et qui coïncide avec $\varphi$ aux points $\underline{x} = x_1 \leq \dots \leq x_k = \bar{x}$.
		\item En déduire un problème linéaire $(P^{in}_k)$ analogue à $(P^{out}_k)$.
		\item Comment déduire de la résolution de $(P^{in}_k)$ deux bornes supérieures de $v_P$. Montrez que l'une des deux est supérieure à l'autre.
	\end{enumerate}
	\item {\em (Pour aller plus loin)} En prenant des points $(x_i)_{i\in[k]}$ équirépartis sur $[\underline{x},\bar{x}]$ calculez, à l'aide de Julia, les 4 bornes précédentes pour $k=10$, $k=50$ et $k=100$. Commentez.
\end{enumerate}


{\em Le rapport est à envoyer par mail à votre chargé de TD pour le 17
Octobre 2017. L'évaluation sera faite sur le rendu complété d'un entretien
avec expérimentation potentielle du modèle (apporter son ordinateur pour
l'entretien).}

\end{document}